\documentclass[../../fulltext/fulltext.tex]{subfiles}
\begin{document}

\chapter{Trying to beat Vegas?  Don't Bet on it}
	\begin{itemize}
		\item How the odds work, your bet changes the odds, because so many knowledgeable people gamble, like the stock market, the Vegas lines tend to be pretty accurate
		\item Your options?  Be better than the market, or you could find a loophole
		\item Talk about Leicester City, the soccer team with 5000-1 odds to win their division$\ldots$ that won their division.
	\end{itemize}
	\section{The Math Behind Gambling}
	Buchanan stuff
	\section{Even Bookmakers make Mistakes}
Prior to the 2016 Premier League season, Leicester City was given 5000-1 odds to end the season with the best record by English bookmakers William Hill and Ladbrokes\footnote{Technically not Vegas, but still a gambling bookmaker}. That means a bet of 10 dollars on Leicester, perhaps a tradition you have followed your whole life, would net you 50,000 dollars.  The reward alone makes the 5000-1 odds even harder to believe.  Unless you believed Leicester City had absolutely no hope of winning the league \footnote{Elvis Presley being found alive had the same odds as Leicester City winning the Premier League in 2016.},\cite{leicester} a 5000-1 odd would make sense to bet on, as even a very small bet (read: low risk) could reel in a substantial reward.  So, the question is, why was Leicester rated so low?

Leicester City Football Club, otherwise known as the Foxes, is a British football roster that, until recently, was a franchise rife with disarray, disappoint, and, at best, mediocrity.  In its 133 year history, Leicester had never won the Premier league, or its predecessor Division One.  Not once.  Even the Cubs, the lowly Chicago Cubs only took 108 years between their championships.  Worse yet, Leicester only finished \emph{second} once, and that was during the 1928-1929 season.\cite{Leicester2}  And, to make matters worse, the Premier League, like many other high level soccer leagues, has been lacking in parity amongst its teams for years on end.  Manchester United, Manchester City, Arsenal, and Chelsea year after year attract top players, and almost always seem to come on top. And finally, to compound the issue, Leicester City came off a miserable 2014-2015 campaign, finishing fourteenth out of twenty teams, only escaping relegation\footnote{Relegation is when teams that finish at the bottom of the Premier league are ``relegated'' to a lower league the next season} with seven wins in their final nine games.\cite{Leicester3}  
The hot finish, however, did not translate to optimism in bookmaker's minds, in part due to skepticism of the talent on the roster and the sacking of manager Nigel Pearson.

Astonishingly, however, Leicester City got off to a hot start in the 2015-2016 season. Striker Jamie Vardy\index{Jamie Vardy}  led his team's tremendous effort in the early season, and Leicester City jumped to the top of the Premier League by Christmas day of 2015\index{leicester5}
.  While many were skeptical at first of the squad's hopes, the team became an international sensation as more and more people became aware of what was going on.  The team with 5000-1 odds was looking more and more likely to win it all.  Instead of tailing off, Leicester maintained their grip on first place.  On May 2, 2016, Leicester City secured the Premier League championship, in the process making some lifelong fans a lot of money, and destroying conventional wisdom in betting. \index{Leicester4}

Unfortunately, lightning did not strike twice, as the club's success was short lived, with the 2016-2017 season falling in line with what was expected of the surprising 2015-2016 campaign. The season ultimately saw the removal of Claudio Ranieri, the (at one point very popular) manager that brought Leicester the title the season before, on February 23, 2017.\index{leicester6}
\begin{table}[t!]
	\centering
	\begin{tabular}{|l||l|}
		\hline
		Event                                     & Odds    \\ \hline
		Leicester winning the Premier League 2016 & 5000-1  \\ \hline
		Hitting a Hole in One                     & 3632-1  \\ \hline
		Fatally Slipping in Shower                & 2232-1  \\ \hline
		Dying as a Pedestrian                     & 672-1   \\ \hline
		Catching Ball at MLB game                 & 570-1   \\ \hline
		Having a Child Genius &261-1 \\ \hline
		Getting on Plane with Drunken Pilot       & 117-1 \\ \hline
	\end{tabular}
	\caption{These are things more likely, statistically to occur, then the odds bookmakers gave Leicester City to win the Premier League title in 2016\cite{leicester7}}
	\label{tab:leic}
\end{table}
\section{On the Average, You'll Lose}

Ostensibly, the conclusion from Leicester City is to bet on the underdog.  Well, not really.  If there is data suggesting that Leicester city was greater than a 5000-1 odd, which they obviously were, then the bet makes a lot of sense.  If your individual odds are different from the bookmakers significantly, it is best to jump in\footnote{Not advocating for gambling, just speaking from a mathematical sense}.  However, people tend to know less than Vegas bookmakers.  While Leicester City is a tremendous story, one that will undoubtedly be told for a while, glorified in the movies, it is an exception.  For example, even the lowly Cleveland Browns, the franchise that's toiled in obscurity for decades, and who had only 3 wins in the previous 2015 NFL season, started the season with 150-1 odds, a number not nearly as drastic as Leicester\cite{leicester}.  


 As evidenced by the data, the best advice, no matter how hot you are, no matter what dream you had, no matter what socks you're wearing, is to quit when ahead.  Many have learned the hard Vegas welcomes avarice as an old friend. The chances of beating the market consistent are slim, and attempting to do so can have catastrophic consequences. 

\end{document}