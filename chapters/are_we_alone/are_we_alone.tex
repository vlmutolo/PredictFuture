\documentclass[../../fulltext/fulltext.tex]{subfiles}
\begin{document}

\begin{savequote}[45mm]
I guess I think of lotteries as a tax on the mathematically challenged.
\qauthor{Roger Jones}
\end{savequote}

For as long as humans have been around we have looked up and wondered whether we are alone. 
And, to date, we do not know the answer to this question.  Granted, there are many, many stories of alien encounters, tales that have become increasingly far fetched over the years, but no proof of extraterrestial life has ever been found. So, despite what many believe, we have no way of knowing for sure whether or not we really are alone in our universe.  Hell, we don't even know if our universe is the only universe\footnote{Many physicists believe in a \enquote{multiverse} theory, which proposes for all intents and purposes an infinite number of universes, and ours is simply one of the many}. So, can we predict when we will meet our terrestial neighbors, if they do indeed exist?  Not really.  But we can predict if there \emph{is} life out there.  Well, kind of.  

How do we go about predicting the existence of extra-terrestial life?  Insert the \textbf{Drake Equation}. We'll begin our story in 1961 at the Green Bank observatory\footnote{ Frank Drake hosted a conference in 1961 on the search for extra-terrestial life.  Green Bank was an ideal gathering for different astronomers.}, a radio telescope located, rather unsurprisingly, in Green Bank, West Virginia.\cite{GreenBank} Green Bank is a town that seems out of place in 2017. For one, the town has a population of 143, and perhaps even more shockingly, the town prohibits the use of cellular devices.  Yes, you read that right.  

The Green Bank observatory was where Frank Drake\index{Frank Drake} presented his now famous Drake equation\index{Drake equation}.  The equation was presented at a conference and intended to answer the ``are we alone'' question from a probabilistic standpoint, providing a range of the number of intelligent alien species in our galaxy, which can more or less be extended to our universe by multiplying by the number of galaxies in the universe\cite{numgalax}\footnote{Estimates put this number in the trillions} Furthermore, the question tried to answer not only if there is \emph{life}, but if there is \emph{intelligent} life.  While this sounds impossible to do with a series of fractions multiplied together, the equation does indeed shed some quantitative light on the question.

In all it's glory, we have reproduced the Drake equation below:
\begin{equation}
\boxed{\boxed{N=R\vdot f_p\vdot n_e\vdot f_\ell\vdot f_i\vdot f_c\vdot L}}
\end{equation}
What do all these variables mean? 
\begin{itemize}
\item 	$N$ is the number of civilizations in our galaxy that are within our ``light cone''\index{light cone}.  That is, the number of civilizations that we would be able to detect if they do exist.  This is an important distinction, as there could very well be civilizations that exist, but if we cannot detect them, we are no closer to our answer than when we started. 
\item $R$ is the rate of formation of stars, or how long it takes the average star to form in our galaxy.  This is important because it typically takes stars billions of years to have the capability of supporting life, and given that the universe is estimated to be around 14 billion years old\cite{universeage}\footnote{13.8 billion years to be exact}, this limits the possibilities of life formation dramatically.
\cite{space.com}  
\item $f_p$ is the fraction of those stars that develop planetary systems, which is an obvious pre-requisite for life (as far we know). Now it is starting to become clearer how the equation works, as we will keep multiplying fractions together to get a final probability, or better a range of probabilities (more on that later).
\item $n_e$ is the number of planets in a solar system with an environment suitable for life.  In our solar system, there is one obvious answer, Earth, but there are speculations that both Mars and one of Saturns's moons, Enceladus\cite{lifenasa}.  Now, if we have life on moons, we have to include this, as nomenclature ultimately does not affect life.
\item $f_\ell$ is the fraction of suitable planets that life actually appears.  In our solar system, out of the potentially habitable planets, only Earth is known to have life (otherwise this whole endeavor would be a waste of time).  
\item $f_i$ is the fraction of life bearing planets on which intelligent life emerges.  There's always one wise guy who'll say zero, but this question is actually quite difficult to answer. Given that there is intelligent life on Earth, our default answer would be 100\%, but since only one or few species out of the billions developed intelligence, that this percentage is extremely low, and in fact close to zero.  We, as humans, tend to fancy ourselves as the only intelligent beings on Earth, and there is a valid argument to be made for that, but many will argue that other animals, such as dolphins and whales\footnote{In case you do not believe this, on the Internet there are videos of whales hunting seals that will probably change your mind}  exhibit signs of intelligence\cite{dolphinssmart}   This will change the probability of intelligent life existing dramatically.
\item $f_c$ is the fraction of civilizations that develop their technology to a point that they can communicate, or at least convey their existence into the galaxy, and thus make themselves detectable.  Again, there could very well be life, but if we cannot see it, it is no good to us.  Perhaps there are alien creatures in our galaxy asking the same questions, unaware of our existence.
\item Finally, $L$ is the length of time that civilizations last, or, more accurately, the length of time they transmit messages into space.  As humans, we have been transmitting our electromagnetic waves\cite{eandm} waves into space since the late 1800's with the invention of the radio.  Now, that means we have been roughly transmitting our existence for about 100 years, give or take.  Therefore, the $L$ for Earth would be about $100$. How long will our civilization last?  Hopefully a while, we can all probably agree on that. Realistically, however, there is a pretty wide range on how long people think we will last, ranging from the very cynical (less than a hundred years) to the very ideal (millions of years). This range again makes the range on the number of civilizations that could exist in our galaxy fairly dramatic, as we shall shortly see.	
\end{itemize} 

Now, comes the tricky (and fun) part; crunching the numbers.  Be warned, the final result will not be a definitive answer as to how many alien civilizations are out there, it will not tell us if they benevolent, nice, maybe even funny good hearted creatures, or cold, malicious monsters hell bent on our destruction.  For that, it is probably best to delve into your science fiction collection and decide for yourself.  Rather, the Drake equation will yield a wide range of possibilities for the number of civilizations.  To do this, we bring in doctor Jeremy Carlo, professor of physics at Villanova, and an amateur astronomer, who has presented on the Drake equation numerous times. With Dr. Carlo, we will go through an example of the Drake equation, and find that, based on different people's beliefs, that it will indeed yield a tremendous range of possibilities concerning the existence of intelligent alien lifeforms in our galaxy.

What the Drake equation essentially tells us then, is that we either are alone in our galaxy, or we are one of countless number of civilizations, with every possibility in between.  So, while it is a fun exercise, the Drake equation is not necessarily telling us anything we did not already know.
\end{document}