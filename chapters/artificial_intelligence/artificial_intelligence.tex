\documentclass[../../fulltext/fulltext.tex]{subfiles}
\begin{document}

\chapter[Rise of the Bots]{Artificial Intelligence: What We can Learn from Tay Bot}
\begin{itemize}
	\item In this chapter, we discuss how artificial intelligence is built off one thing: \emph{Data}, and lots of it.
	\item Talk about how machine learning using artificial networks is really a means of predicting how humans act, and if this can be emulated by computers, essentially simulating intelligence
\end{itemize}
Artificial intelligence.  The singularity.  Keanu Reeves.  Where to start?  What is artificial intelligence?  Are we doomed?  Aren't computers already smart?  Can we live forever?  When is this going to happen? These are questions that are commonly associated with artificial intelligence, and while we will certainly answer them as well as we can without diving too deep into philosophical waters, we would also like to explain some of the underlying theory behind machine learning, and ultimately how it connects to predictive analytics \footnote{In a nutshell, artificial intelligence is predictive analytics}.  

Whenever artificial intelligence is mentioned, a sense of fear tends to take hold of people.  Perhaps influenced by popular movies such as the Terminator, or the Matrix, where our former friendly computers realize humans make better batteries than conversation, there is a legitimate fear in many about a one-day \textbf{singularity} \index{singularity}. What is the singularity?  Broadly speaking, it is when computers will become sentient.  By this interpretation\footnote{By no means the only, or even the correct interpretation}  a computer will be cognizant of its existence.  Why does this matter?  Well, computers, with their incredible speed, processing power, and memory dwarf humans in computational ability.  But, until recently, they could not think for themselves.  They could not create.  Computers were programmable, and thus only as smart as their creators made them.  Recently, however, computers have been able to ``learn'' from their mistakes, and exhibit signs of intelligence. 

Eventually, these computers will think like humans. That is, they will think like humans with nearly infinite knowledge and processing speed that not even transcendental geniuses the likes of Isaac Newton could not even fathom. Eventually, the computers will move beyond their programming, or so it is theorized. In theory, computers will be able to accelerate their learning at incomprehensible speeds, due to the processing power and memory capabilities they are endowed, and technology will increase so rapidly that the human mind cannot imagine where it will lead.  This belief is more technically what is meant by \emph{the singularity}, and it is this definition that is often accompanied by fear and bewilderment among people\footnote{Well known scientists have expressed their reservations, to put it mildly, about artificial intelligence, such as acclaimed theoretical physicist Stephen Hawking\index{Stephen Hawking}}, thanks at least in part to doomsday predictions in movies, books, television and so on. But is their fear justified?  Will we meet our fiery doom at the hands of our metallic counterparts.  Well, in accordance with the central theme of this book, \emph{probably} not.  

\end{document}